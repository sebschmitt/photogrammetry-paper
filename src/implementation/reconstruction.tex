\section{Rekonstruktion}


In diesem Abschnitt werden die einzeln Schritte zum Rekonstruieren der Korrespondenzen erklärt, die im vorherigen Prozess gebildet worden sind. 
Für die Rekonstruktion wird die Klasse \emph{SceneReconstructor} in der Datei \emph{reconstruction.hpp} definiert und in \emph{reconstruction.cpp} implementiert. 
Die Klasse besitzt einen Konstruktor, welche eine Instanz der Calibration Klasse als Argument benötigt. 
Mit der Instanz wird die private Member-Variable \emph{calbiration} instanziiert, so dass sie später bei manchen Berechnungen verwendet werden kann.
Des Weiteren wird öffentlich die Methode \emph{reconstructScenes} definiert.
Diese Methode erfordert eine Instanz der Iterator-Klasse mit dem ImagePair als Template-Type. %TODO: check spelling
Mit dem Iterator werden in der Methode die korrespondierenden Bildpunkte rekonstruiert.
Dabei wird darauf geachtet, dass die rekonstruierten Punkte der einzelnen Paare später auch zusammen passen.
Wie in XXX zusehen ist, werden einige weitere private Methoden für die Klasse definiert.
Diese werden in den folgend Paragraphen erläutert, wenn sie für bestimmte Berechnungen genutzt werden.

Die Rekonstruktion wird mit den folgenden 4 Schritten durchgeführt:

\begin{enumerate}
    \item Lokale Rotation \& Translation bestimmen
    \item globale Translation berechnen
    \item Projektionsmatrix berechnen
    \item Triangulation
    \item Skalieren
        \begin{enumerate}
            \item skalierte globale Translation berechnen
            \item Projektionsmatrix berechnen
            \item Triangulation
        \end{enumerate}
\end{enumerate}




