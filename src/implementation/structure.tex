\section{Projektstruktur}\ohead{Sebastian Schmitt}
Die Applikation wird in C++ entwickelt.
Auch wenn Performance kein Ziel ist, wurde C++ gewählt um eine höhere Performance zu erreichen.
Einzige Abhängigkeit ist die offene Bibliothek OpenCV, siehe dazu \ref{sec:opencv}.
Zur Buildsteuerung wird das quelloffene Tool CMake\footnote{https://cmake.org/} verwendet.

Das Projekt ist folgendermaßen strukturiert:

\dirtree{%
.1 yapgt.
.2 include \ldots{} \begin{minipage}[t]{12cm}
Header-Dateien
\end{minipage}.
.2 resources \ldots{} \begin{minipage}[t]{12cm}
Bilder zum Testen von Kalibrierung und Feature-Matching
\end{minipage}.
.2 src \ldots{} \begin{minipage}[t]{12cm}
Implementierung
\end{minipage}.
.2 CMakeLists.txt \ldots{} \begin{minipage}[t]{12cm}
CMake Konfiguration
\end{minipage}.
}
