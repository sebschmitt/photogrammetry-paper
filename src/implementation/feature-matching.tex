\section{Feature Matching}\ohead{Sebastian Schmitt}
Zum Matchen der im vorherigen Schritt identifizierten Features wird ein Bruteforce Matcher verwendet.
Die Entscheidung für den Bruteforce Matcher viel, da er einfach zu verwenden und dabei gründlich ist.
Die geringere Performance ist für diese Arbeit nicht wichtig.

Das Feature Matching ist ebenfalls in der Methode \emph{generateSequence} der Klasse \emph{SequenceMatcher} implementiert.
Nachdem für alle Bilder Features identifiziert und Deskriptoren gebildet wurden, werden letztere paarweise abgeglichen.
Dazu werden Paare gebildet welche immer aus einem Bild und dem direkten Nachfolger bestehen.
So entsteht bspw. aus Bild 1 und Bild 2, aus Bild 2 und Bild 3 sowie aus Bild 3 und Bild 4 jeweils ein Paar.
Zum Matchen wird OpenCVs Bruteforce Matcher \emph{cv::BFMatcher}\cite{opencv_doc_bfmatcher} verwendet.
Dieser wird mit dem Parameter \emph{cv::NORM\_L2} initisalisiert, um \emph{L2} als Normierungstyp zu verwenden.
\emph{L1} und \emph{L2} sind dabei die empfohlenen Normierungstypen für SIFT \cite{opencv_doc_bfmatcher_create}.
Da bei Tests mit \emph{L2} bessere Matches gefunden werden konnten als mit \emph{L1}, wird \emph{L2} verwendet.
Zusätzlich wird die Option \emph{crossCheck} des Bruteforce Matcher aktiviert.
Dadurch wird verhindert, dass ein Keypoint mit mehreren Keypoints abgeglichen wird \cite{opencv_doc_bfmatcher_create}.
Für jedes Bildpaar wird nun die Funktion \emph{match}\cite{opencv_doc_match} des Bruteforce Matchers aufgerufen.
Übergeben werden jeweils die Feature-Deskriptoren sowie ein Pointer zu einem \emph{std::vector<cv::DMatch>} in welchem die gefundenen Matches gespeichert werden.
Danach wird für jedes Paar basierend auf den Bildpunkten eine Fundamental Matrix gebildet.
Dazu wird OpenCVs \emph{cv::findFundamentalMatrix}\cite{opencv_doc_find_fundamental_mat} verwendet.
Danach werden solche Keypoints herausgefiltert, welche nicht Teil der Maske sind.
Falls vom Nutzer entsprechend so eingestellt wird zum Schluss für jedes Paar ein Bild gespeichert in welchem die gefundenen Matches visualisert sind.
Ein solches Bild ist in \autoref{fig:example-match-output} zu sehen.