\section{Feature Identifizierung}\label{sec:feature-identification}
\ohead{Sebastian Schmitt}
Zur Feature Identifizierung wird der Algorithmus \emph{Scale-Invariant Feature Transform} (SIFT) verwendet.
SIFT bietet den Vorteil dass viele Keypoints gefunden werden und der Algorithmus, in gewissen Grenzen, invariant gegenüber Transformationen wie Translation, Rotation und Skalierung ist.
Zudem bietet SIFT gegenüber Weiterentwicklungen wie Speeded Up Robust Features (SURF) den Vorteil, dass das Patent mittlerweile ausgelaufen ist und der Algorithmus jetzt frei verwendet werden kann.

Die Feature Identifizierung findet in der Methode \emph{generateSequence} der Klasse \emph{SequenceMatcher} statt.
Hierzu werden für jedes Bild folgende Schritte durchgeführt:\\
Zunächst wird ein ImageContainer-Struktur erzeugt.
Danach wird das Bild mittels \emph{cv::imread} als \emph{cv::Mat} eingelesen.
Diese Matrix wird nun mittels \emph{resize} um den Downsampling Faktor herunterskaliert um die Bearbeitung zu beschleunigen.
Danach wird die Methode \emph{undistortImage} des übergebenen \emph{Calibration} Objekts verwendet um das Bild zu entzerren.
Danach wird eine Instanz der OpenCV Implementierung von SIFT erzeugt.
Dabei werden als Parameter für \emph{nFeatures} 0, um alle Features zu erhalten, für \emph{nOctaveLayers} 3, für \emph{contrastThreshold} 0.04, für \emph{edgeThreshold} 10 sowie für \emph{sigma} 1.6 übergeben.
Zuletzt wird die Funktion \emph{detectAndCompute} der SIFT Instanz aufgerufen.
Als Parameter wird hier das eingelesene Bild, ein Leeres Array (\emph{cv::noArray}) als Maske sowie Pointer für die Rückgabe der Keypoints sowie der Deskriptoren verwendet.