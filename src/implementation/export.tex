\section{Export}\ohead{Sebastian Schmitt}
Die rekonstruierten Punkte können exportiert und so in 3D-Animationsprogrammen wie Blender\footnote{https://blender.org} weiterverarbeitet werden.
Dazu wird das Polygon File Format (PLY) verwendet, welches in \autoref{sec:ply-format} genauer erklärt wird.

Die Schnittstelle zum Export ist in der abstrakten Klasse \emph{ModelExporter}, welche in \emph{include/modelexporter.hpp} spezifiziert ist, definiert.
Ziel dieser Lösung ist es mehrere, austauschbare Exporter zu verwenden, wenngleich aktuell nur der PLY Exporter implementiert ist.
Dieser ist in der Klasse \emph{PlyModelExporter} implementiert, welche von \emph{ModelExporter} erbt.
Diese Klasse ist in \emph{include/model-exporters.hpp} spezifiziert und in \emph{src/ply-model-exporter.cpp} implementiert.


\subsection{Polygon File Format}\label{sec:ply-format}
Die Datei beginnt mit einem Header.
In diesem wird der Aufbau der Datei genauer spezifiziert.
Der Header startet dabei immer mit der magischen Zahl \emph{ply}.
In der zweiten Zeile wird mit \emph{format} das verwendete Format sowie die Version des Formats festgelegt.
Verfügbare Formate sind entweder \emph{ascii}, \emph{binary\_little\_endian} sowie \emph{binary\_big\_endian}.
Aktuell gibt es nur Version 1.0.
In dieser Arbeit wird zur Ausgabe das \emph{ascii} Format verwendet.
Auch wenn ein Binärformat gewählt wird, so besteht der Header immer aus ASCII-Zeichen.
Mit dem \emph{element} Keyword beginnt die Beschreibung der verwendeten Elemente sowie deren Anzahl.
Mit \emph{property} können verwendete Eigenschaften der Elemente sowie deren Datentypen festgelegt werden.
In dieser Arbeit werden lediglich \emph{vertex} Elemente verwendet.
Diese haben die Attribute \emph{x}, \emph{y} und \emph{z} sowie \emph{red}, \emph{green} und \emph{blue}.
Eine Beispieldefinition ist in \autoref{lst:ply_vertex_element} zu sehen.
Der Header wird mit dem Keyword \emph{end\_header} beendet.
Darauf folgen alle Elemente in der im Header festgelegten Reihenfolge.
Es ist also nicht nötig zusätzlich den Typ des Elements anzugeben.
Um Beispielsweise einen roten Punkt mit den Koordinaten $\left(1, 2, 3\right)$ zu definieren, braucht man folgenden Eintrag:\\
\emph{1.0	2.0	3.0	255	0	0}

\begin{lstlisting}[numbers=left, breaklines=true, breakatwhitespace=false, label=lst:ply_vertex_element, caption=Beispiel PLY Header für 10 Vertex Elemente mit den Attributen Position und Farbe]
element vertex 10
property float x
property float y
property float z
property uchar red
property uchar green
property uchar blue
end_header
\end{lstlisting}

\subsection{Farben}
Zur Bestimmung der Farben der Punkte gibt es die Klasse \emph{ColorPicker} welche in \emph{include/colors.hpp} spezifiziert und in \emph{src/colors.cpp} implementiert ist.
Diese stelle eine öffentliche, statische Methode bereit welche als Parameter zwei Bilder, zwei Punkte sowie einen Radius annimmt.
Zunächst wird pro Bild über den Radius um den übergebenen Punkt herum die Durchschnittsfarbe bestimmt, danach wird der Durchschnitt beider Farben gebildet.

Leider kann Blender nicht von sich aus Farben in Punktwolken visualisieren.
Möchte man in Blender auch die Farben darstellen, so bietet sich das Addon \emph{Point Cloud Visualizer}\footnote{https://blenderartists.org/t/point-cloud-visualizer/684180} an.
Zur Installation muss lediglich die Datei \emph{space\_view3d\_point\_cloud\_visualizer.py} in den Blender Add-on Einstellungen installiert werden.
Zur Verwendung kann man der 3D View mittels \emph{Add}  $\,\to\,$ \emph{Mesh} $\,\to\,$ \emph{Plane} ein Plane hinzufügen.
Durch drücken der Taste \emph{n} öffnet sich rechts ein Optionsmenü.
Dieses hat als zusätzlichen Punkt nun \emph{Point Cloud Visualizer}.
Hier kann die PLY Datei geladen und gerendert werden\cite{blender_point_cloud_visualizer}.