\section{OpenCV}\label{sec:opencv}\ohead{Sebastian Schmitt}
Es wird OpenCV in der Version 4.2.0 mit dem Modul OpenCV Contrib verwendet.
OpenCV Contrib beinhaltet OpenCV Module, welche jedoch nicht stabil genug sind um Teil der offiziellen OpenCV Distribution zu sein \cite[README.md]{opencv_2013}.
OpenCV wurde mit folgenden Optionen kompiliert:
\begin{itemize}
	\item OPENCV\_EXTRA\_MODULES\_PATH um den Pfad zu OpenCV Contrib festzulegen
	\item OPENCV\_ENABLE\_NONFREE um auch nicht freie Algorithmen zu kompilieren
\end{itemize}
Auch wenn es Anforderung ist nur freie Bibliotheken zu verwenden, ist der Flag OPENCV\_ENABLE\_NON\_FREE notwendig.
Der in dieser Arbeit verwendete Algorithmus Scale-Invariant Feature Transform (SIFT) war bis 06. März 2020 von der University of British Columbia patentiert\footnote{siehe dazu https://patents.google.com/patent/US6711293} und ist daher in der verwendeten OpenCV Version nur mit OPENCV\_ENABLE\_NON\_FREE nutzbar.
SIFT ist Teil des Moduls Features2D extra (xfeatures2d) welches Teil von OpenCV Contrib ist und sowohl experimentelle als auch nicht freie Algorithmen beinhaltet.

Um die Applikation auch nutzen zu können ohne vorab OpenCV zu installieren oder zu kompilieren wurde CMake angewiesen, OpenCV statisch zu linken (siehe \autoref{lst:cmake_opencv_static}, Zeilen 1-4).
Zusätzlich kopiert CMake alle gelinkten Bibliotheken, auch wenn diesen nicht referenziert werden, in den Ausgabeordner, damit diese nicht von Hand kopiert werden müssen (\autoref{lst:cmake_opencv_static}, Zeilen 6-8).

In der \autoref{tab:opencv-datatypes} werden alle verwendeten Datentypen von OpenCV aufgelistet und beschrieben.

\begin{lstlisting}[numbers=left, breaklines=true,breakatwhitespace=false,label=lst:cmake_opencv_static, caption=Ausschnitt von CMakeLists.txt um OpenCV statisch zu linken]
set(OPENCV_STATIC ON)
find_package(OpenCV REQUIRED)
[..]
target_link_libraries(yapgt ${OpenCV_LIBS})
[..]
foreach(CVLib ${OpenCV_LIBS})
    file(COPY ${_OpenCV_LIB_PATH}/${CVLib}${OpenCV_VERSION_MAJOR}${OpenCV_VERSION_MINOR}${OpenCV_VERSION_PATCH}d.dll DESTINATION ${CMAKE_BINARY_DIR})
endforeach()
\end{lstlisting}


\begin{table}[]
\begin{tabularx}{\textwidth}{ |l|X| }
\textbf{Datentyp}     & \textbf{Beschreibung}                                                                                                                                               \\ \hline
cv.:DMatch            & Beinhaltet übereinstimmende Deskriptoren \cite{opencv_doc_dmatch}                                                                                \\
cv::Exception         & Allgemein Exception Klasse für OpenCV Exceptions \cite{opencv_doc_exception}                                                                     \\
cv::FileStorage       & Utility Klasse zum Speichern und Laden von XML/YAML/JSON Dateien \cite{opencv_doc_file_storage}                                                 \\
cv::KeyPoint          & Datenstruktur zur Beschreibung von Keypoints \cite{opencv_doc_keypoint}                                                                          \\
cv::Mat               & n-dimensionales Array, kann zur Repräsentierung von Matrzien, Bildern, Vektoren, Punktwolken, und weitere verwendet werden \cite{opencv_doc_mat} \\
cv::Mat1d             & cv::Mat vom Typ double \cite{opencv_doc_mat1d}                                                                                                   \\
cv::Mat1f             & cv::Mat vom Typ float \cite{opencv_doc_mat1f}                                                                                                    \\
cv::Point2f           & Repräsentiert einen zweidimensionalen Punkt, die Koordinaten werden als float repräsentiert \cite{opencv_doc_point2f}                            \\
cv::Point3f           & Repräsentiert einen dreidimensionalen Punkt, die Koordinaten werden als float repräsentiert \cite{opencv_doc_point3f}                            \\
cv::Ptr               & Smarter Zeiger \cite{opencv_doc_ptr}                                                                                                             \\
cv::Rect              & Repräsentiert ein Rechteck, die Koordinaten, sowie Länge und Breite werden als Integer repräsentiert \cite{opencv_doc_rect}                      \\
cv::Size              & Hält Breite und Höhe, welche als Integer repräsentiert werden \cite{opencv_doc_size}                                                             \\
cv::Vec3b             & Dreidimensionaler Vektor, Komponenten werden als uchar repräsentiert \cite{opencv_doc_vec3b}                                                     \\
cv::xfeatures2d::SIFT & Klasse welche den SIFT Algorithmus implementiert \cite{opencv_doc_sift}                                                                         
\end{tabularx}
\caption{Diese Tabelle bietet eine Auflistung aller verwendeten OpenCV Datentypen}
\label{tab:opencv-datatypes}
\end{table}