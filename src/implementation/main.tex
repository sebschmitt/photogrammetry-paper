% !TeX root = ../../main.tex
\ohead{Sebastian Schmitt}

\chapter{Umsetzung}
Im folgenden wird der Aufbau des Projektes sowie verwendete Bibliotheken erläutert.
Zusätzlich wird auf die Implementierung der einzelnen Schritte der SfM Pipeline eingegangen.

\section{Projektstruktur}
Die Applikation wird in C++ entwickelt.
Auch wenn Performance kein Ziel ist, wurde C++ gewählt um eine höhere Performance zu erreichen.
Einzige Abhängigkeit ist die offene Bibliothek OpenCV, siehe dazu \ref{sec:opencv}.
Zur Buildsteuerung wird das quelloffene Tool CMake verwendet.

Das Projekt ist folgendermaßen strukturiert:

\dirtree{%
.1 yapgt.
.2 include \ldots{} \begin{minipage}[t]{12cm}
Header-Dateien
\end{minipage}.
.2 resources \ldots{} \begin{minipage}[t]{12cm}
Bilder zum Testen von Kalibrierung und Feature-Matching
\end{minipage}.
.2 src \ldots{} \begin{minipage}[t]{12cm}
Implementierung
\end{minipage}.
.2 CMakeLists.txt \ldots{} \begin{minipage}[t]{12cm}
CMake Konfiguration
\end{minipage}.
}


\subsection{OpenCV}\label{sec:opencv}
Es wird OpenCV in der Version 4.2.0 mit dem Modul OpenCV Contrib verwendet.
OpenCV Contrib beinhaltet OpenCV Module, welche jedoch nicht stabil genug sind um Teil der offiziellen OpenCV Distribution zu sein \cite[README.md]{opencv_2013}.
OpenCV wurde mit folgenden Optionen kompiliert:
\begin{itemize}
	\item OPENCV\_EXTRA\_MODULES\_PATH um den Pfad zu OpenCV Contrib festzulegen
	\item OPENCV\_ENABLE\_NONFREE um auch nicht freie Algorithmen zu kompilieren
\end{itemize}
Auch wenn es Anforderung ist nur freie Bibliotheken zu verwenden, ist der Flag OPENCV\_ENABLE\_NON\_FREE notwendig.
Der in dieser Arbeit verwendete Algorithmus Scale-Invariant Feature Transform (SIFT) war bis 06. März 2020 von der University of British Columbia patentiert\footnote{siehe dazu https://patents.google.com/patent/US6711293} und ist daher in der verwendeten OpenCV Version nur mit OPENCV\_ENABLE\_NON\_FREE nutzbar.
SIFT ist teil des Modules Features2D extra (xfeatures2d) welches Teil von OpenCV Contrib ist und sowohl experimentelle als auch nicht freie Algorithmen beinhaltet.

Um die Applikation auch nutzen zu können ohne vorab OpenCV zu installieren oder zu kompilieren wurde CMake angewiesen, OpenCV statisch zu linken (siehe \autoref{lst:cmake_opencv_static}, Zeilen 1-4).
Zusätzlich kopiert CMake alle gelinkten Bibliotheken, auch wenn diesen nicht referenziert werden, in den Ausgabeordner, damit diese nicht von Hand kopiert werden müssen (\autoref{lst:cmake_opencv_static}, Zeilen 6-8).

\begin{lstlisting}[numbers=left, breaklines=true,breakatwhitespace=false,label=lst:cmake_opencv_static, caption=Ausschnitt von CMakeLists.txt um OpenCV statisch zu linken]
set(OPENCV_STATIC ON)
find_package(OpenCV REQUIRED)
[..]
target_link_libraries(yapgt ${OpenCV_LIBS})
[..]
foreach(CVLib ${OpenCV_LIBS})
    file(COPY ${_OpenCV_LIB_PATH}/${CVLib}${OpenCV_VERSION_MAJOR}${OpenCV_VERSION_MINOR}${OpenCV_VERSION_PATCH}d.dll DESTINATION ${CMAKE_BINARY_DIR})
endforeach()
\end{lstlisting}


\section{Kommandozeilenargumente}

\section{Kamera Kalibrierung}

\section{Feature Identifizierung}

\section{Feature Matching}

\section{Rekonstruktion}

\section{Export}
\subsection{PLY Format}
\subsection{Farben}