\section{Kamera Kalibrierung}\label{sec:implementation-calibration}
Die Kalibrierung der Kamera wird von der Klasse \emph{Calibration} durchgeführt.
\emph{KameraMatrix}, \emph{OptimalMatrix} sowie \emph{DistoritonCoefficients} werden in selbiger vorgehalten, so dass einfach ein \emph{Calibration} Objekt an die weiteren Schritte der Pipeline weitergeleitet werden.
Die Klasse \emph{Calibration} ist in \emph{include/calibration.hpp} spezifiziert und in \emph{src/calibration.cpp} implementiert.
Der Konstruktor benötigt keine Argumente.
Die Klasse stellt folgende Methoden zur Verfügung:
\begin{itemize}
\item \emph{cv::Mat\& getCameraMatrix()} gibt die Kamera-Matrix zurück
\item \emph{void calibrate(std::vector<std::filesystem::path> imageFiles, cv::Size boardSize)} führt die eigentliche Kalibrierung der Kamera anhand der übergebenen Bilder durch, siehe dazu \autoref{sec:calibration-calibration}
\item \emph{void loadCalibration(std::filesytem::path filepath)} Lädt die Kalibrierung aus der angegebenen Datei, siehe dazu \autoref{sec:calibration-load-save}
\item \emph{void saveCalibration(std::filesystem::path filepath)} Speichert die Kalibrierung in der angegebenen Datei, siehe dazu \autoref{sec:calibration-load-save}
\item \emph{void undistortImage(const cv::Mat\& image, cv::Mat\& undistoredImage)} entzerrt das übergebene Bild und speichert das entzerrte Bild in \emph{undistoredImage}
\end{itemize}

\subsection{Kalibrierung}\label{sec:calibration-calibration}
Zunächst wird ein Vektor aus Punkten erzeugt welcher die Koordinaten aller Schachbrettfelder hält.
Die Z-Komponente des Vektors ist dabei immer 0.
Es wird davon ausgegangen, dass das Schachbrettmuster auf allen Bildern vollständig zu sehen ist.
Deshalb kann dieser Vektor später für jedes Bild verwendet werden.

Danach wird jedes Bild geladen und mit dem Faktor zwei herunter skaliert um die Berechnungen zu beschleunigen.
Dafür wird OpenCVs \emph{resize}\cite{opencv_doc_resize} Funktion verwendet.
Danach wird das Bild in ein Graustufenbild konvertiert\cite{opencv_doc_cvt_color}.
Dann wird OpenCVs \emph{findChessboardCorners}\cite{opencv_doc_find_chessboard_corners} aufgerufen.
Als Parameter werden das Bild, die Dimension des Schachbretts, einen Vektor von Punkten, in welchem die gefundenen Ecken gespeichert werden, sowie die Flags \emph{CALIB\_CB\_ADAPTIVE\_THRESH} und \emph{CALIB\_CB\_FILTER\_QUADS} verwendet.
\sloppy
\emph{CALIB\_CB\_ADAPTIVE\_THRESH} sorgt dafür, dass adaptive Schwellenwerte verwendet werden sodass die Kacheln des Schachbretts nicht vollkommen schwarz bzw. weiß sein muss.
Es wird, entgegen des Standardverhalten, kein Schwellwert anhand der durchschnittlichen Helligkeit verwendet\cite{opencv_doc_find_chessboard_corners}.
Nach dem setzen des Schwellwerts versucht der Algorithmus Vierecke zu lokalisieren.
Wenn \emph{CALIB\_CB\_FILTER\_QUADS} gesetzt ist, werden zusätzlich weitere Einschränkungen getroffen um falsche Vierecke auszuschließen\cite{opencv_doc_find_chessboard_corners}.
Falls \emph{findChessboardCorners} ein Schachbrettmuster erkannt hat, wird die Präzision der gefundenen Ecken mittels \emph{cornerSubPix}\cite{opencv_doc_corner_sub_pix} erhöht.
Auch wenn \emph{cornerSubPix} von \emph{findChessboardCorners} automatisch aufgerufen wird, so kann die Präzision noch weiter erhöht werden, in dem \emph{cornerSubPix} erneut mit enger gewählten Parametern aufgerufen wird.

Nachdem dieser Prozess für jedes Bild durchgeführt wurde, wird die eigentliche Kalibrierung durchgeführt.
Dafür wird OpenCVs \emph{calibrateCamera}\cite{opencv_doc_calibrate_camera} verwendet.
Als Parameter werden ein Vektor von Vektoren der einzelnen, gefundenen Ecken, ein Vektor von Vektoren der Schachbrett Koordinaten, die Größe der Bilder sowie Variablen für Rückgabe von Kamera-Matrix, Verzerrungskoeffizienten, Rotationsvektor sowie Translationsvektor übergeben.



\subsection{Speichern und Laden}\label{sec:calibration-load-save}
Zum Speichern und zum Laden wird OpenCVs \emph{FileStorage}\cite{opencv_doc_file_storage} verwendet.
Dieser unterstützt die Dateiformate \emph{XML}, \emph{YAML} sowie \emph{JSON}.
Konkret gespeichert werden die Matrizen \emph{cameraMatrix}, \emph{optimalCameraMatrix} sowie \emph{distortionCoefficients}.
Für jede dieser Variablen wurde eine Konstante definiert welche dem FileStorage als Schlüssel dient.
In \autoref{lst:opencv_filestorage} ist zu sehen, wie mit einem \emph{FileStorage} Objekt gearbeitet werden kann.
Bei \emph{cameraMatrixSerName} sowie \emph{distortionCoefficientsSerName} handelt es sich dabei um die konstanten Schlüssel unter welchen die Daten abgespeichert werden.
Zu beachten ist, dass beim Erzeugen des \emph{FileStorage} Objekts kein Format angegeben wurde.
Das zu verwendende Format wird anhand der Dateiendung bestimmt.

\begin{lstlisting}[language=c++, numbers=left, breaklines=true, breakatwhitespace=false, label=lst:opencv_filestorage, caption=Schreiben mit OpenCV Filestorage]
cv::FileStorage fs(filepath.string(), cv::FileStorage::WRITE);
fs << cameraMatrixSerName << cameraMatrix;
fs << distortionCoefficientsSerName << distortionCoefficients;
fs.release();
\end{lstlisting}

Analog zum Speichern kann \emph{FileStorage} zum Lesen verwendet werden (siehe \autoref{lst:opencv_filestorage_read}.
Dafür kann man auf das \emph{FileStorage} Objekt wie auf ein assoziatives Array zugreifen (Zeile 2).
Zusätzlich kann dann der Typ des Rückgabewerts festgelegt werden.
In diesem Beispiel wird der unter dem Schlüssel \emph{cameraMatrixSerName} gespeicherte Wert als OpenCV Matrix geliefert.

\begin{lstlisting}[language=c++, numbers=left, breaklines=true, breakatwhitespace=false, label=lst:opencv_filestorage_read, caption=Lesen mit OpenCV Filestorage]
FileStorage fs(filepath.string(), FileStorage::READ);
cameraMatrix = fs[cameraMatrixSerName].mat();
distortionCoefficients = fs[distortionCoefficientsSerName].mat();
fs.release();
\end{lstlisting}


Eine von \emph{FileStorage} erzeugte XML Datei ist in \autoref{appendix:opencv_filestorage_example} \nameref{appendix:opencv_filestorage_example} zu sehen.