\section{Kommandozeilenargumente}\label{sec:cli-args}
Die einzelnen Schritte der Pipeline werden mittels Kommandozeilenargumente konfiguriert.
Zum Parsen der übergebenen Argumente wurde ein Argumentparser geschrieben.
Dieser akzeptiert Argumente in der Form \emph{-Argumentname}, wobei es egal ist wie viele Bindestriche vor dem Argumentnamen stehen.
Da nach jedem Bindestrich ein neues Argument beginnt, ist es nicht möglich Bindestriche innerhalb von Argumentnamen zu verwenden.
Dieser ist in \emph{include/argparser.hpp} spezifiziert und in \emph{src/argparser.c} implementiert.
Er beinhaltet den Namespace \emph{argparser} sowie zwei Klassen:
\begin{itemize}
\item \textbf{Argument}, repräsentiert ein Arguments (Name, Beschreibung, ob es übergeben wurde, welcher Wert übergeben wurde, ob es übergeben werden muss)
\item \textbf{ArgumentParser}, kennt alle Argumente, überprüft die übergebenen Argumente und legt fest ob und wie ein Argument übergeben wurde
\end{itemize}

Damit der \emph{ArgumentParser} auf Attribute von \emph{Argumenten} zugreifen darf, welche nicht öffentlich sein sollen, wurden diese als \emph{protected} markiert und \emph{ArgumentParser} innerhalb der Klasse \emph{Argument} als \emph{Friend Class} markiert (siehe dazu \autoref{lst:argparser_argument_friend}, Zeilen 2-3).
Auch wenn der übergebene Wert intern als String repräsentiert wird, kann er mit der Funktion \emph{getValue()} auch in andere primitive Typen konvertiert werden (\autoref{lst:argparser_argument_friend}, Zeilen 7-12).

\begin{lstlisting}[language=c++, numbers=left, breaklines=true, breakatwhitespace=false, label=lst:argparser_argument_friend]
namespace argparser {
    class Argument {
        friend class ArgumentParser;

    public:
        [..]
        template<typename T> T getValue() {
            std::istringstream in(this->value);
            T t = T();
            in >> t >> std::ws;
            return t;
        }
        [..]
    protected:
        std::string value;
        [..]
    };
    [..]
}
\end{lstlisting}

\begin{lstlisting}[language=c++, numbers=left, breaklines=true, breakatwhitespace=false, label=lst:argparser_example, caption=Argparser Verwendung]
int main(int argc, char *argv[]) {
    argparser::ArgumentParser parser("yapgt (yet another photogrammetry tool)");
    argparser::Argument a_loadCalibration("loadCalibration", "Filepath to load calibration from");    
    parser.addArgument(&a_loadCalibration);
    parser.parseArguments(argc, argv);
    if (a_loadCalibration.isFound()) {
        filesystem::path path(a_loadCalibration.getValue<string>());
    }
}
\end{lstlisting}

Der Argumentparser kann wie in \autoref{lst:argparser_example} zu sehen ist verwendet werden:\\
Zunächst wird eine Instanz von \emph{ArgumentParser} erzeugt.
Als Parameter wird der Applikationsname übergeben.
Dieser wird beim Aufruf der Applikation mit dem Argument \emph{-help} angezeigt.
Danach wird ein Argument erzeugt.
Als Parameter werden der Name des Argumentes, sowie eine Beschreibung für Hilfe übergeben.
Danach wird das Argument dem ArgumentParser hinzugefügt.
Mit der Funktion \emph{parseArguments(argc, argv)} wird das eigentliche Verarbeiten der Argumente durchgeführt.
Mittels der Funktion \emph{isFound()} wird nun überprüft, ob das Argument übergeben wurde.
Der übergebene Wert kann nun mittels \emph{getValue<string>()} als String weiter verarbeitet werden.

In \autoref{tab:arguments} sind alle verfügbaren Argumente so wie eine Beschreibung aufgelistet.

\begin{table}[]
\centering
\begin{tabularx}{\textwidth}{ lX }
\hline
Name              & Beschreibung                                                                                                              \\ \hline
loadCalibration   & Dateipfad, aus dem die Kalibrierung geladen wird, kann nicht gleichzeitig mit calibrationImages werden                    \\
saveCalibration   & Dateipfad, in dem die Kalibrierung gespeichert wird                                                                       \\
calibrationImages & Ordnerpfad, Bilder welche zur Kalibrierung verwendet werden, kann nicht gleichzeitig mit loadCalibration verwendet werden \\
calibrateRow      & Anzahl der Zeilen des Kalibrierungsmusters                                                                                \\
calibrateColumn   & Anzahl der Spalten des Kalibrierungsmusters                                                                               \\
matchImages       & Ordnerpfad, Bilder, welche zum Matching verwendet werden                                                                  \\
out               & Dateipfad, in den die Punkte exportiert werden                                                                            \\
matchOutput       & Ordnerpfad, visueller Export aller Matches                                                                                \\ \hline
\end{tabularx}
\caption{Übersicht über alle verfügbaren Argumente}
\label{tab:arguments}
\end{table}