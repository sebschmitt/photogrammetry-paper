% !TeX root = ../../main.tex

\chapter{Stand der Technik}

\section{Was ist Photogrammetrie?}
Mit Photogrammetrie bezeichnet man Techniken, mit denen man Informationen wie Position, Orientierung, Form und Größe eines Objekts aus Bildern rekonstruieren kann.
Dabei unterscheidet man ob es sich bei den Bildern um photochemische (konventionelle Photographie), photoelektrische (digitale Photographie) oder um Laserscanner-Aufnahmen, dabei handelt es sich um Bilder welche zusätzlich Entfernungsinformationen zu jedem Bildelement enthalten, handelt.
Je nach Art der Weiterverarbeitung sowie Aufnahme der Bilder wird der Begriff der Photogrammetrie weiter differenziert.
Bei Weiterverarbeitung von photochemischen Bildern mittels optisch-mechanischer Geräte spricht man von \textbf{analoger Photogrammetrie}.
Bei Weiterverarbeitung von photochemischen Bildern mittels Computer spricht man von \textbf{analytischer Photogrammetrie}.
Bei Weiterverarbeitung von photoelektrischen Bildern mittels Computer, ein voll digitaler Prozess, spricht man von \textbf{digitaler Photogrammetrie}.
Die Ergebnisse einer Auswertung mittels Photogrammetrie können sein: \cite{kraus_2004}
\begin{itemize}
\item \textbf{Maßzahlen}, Koordinaten einzelner Objektpunkte in einem dreidimensionalen Koordinatensystem
\item \textbf{Zeichnungen}, Karten und Pläne im Grundriss mit Höhenlinien
\item \textbf{geometrische Modelle}
\item \textbf{Bilder}, entzerrte Fotos, Luftbildkarten
\end{itemize}

In dieser Arbeit werden Techniken der digitalen Photogrammetrie verwendet mit dem Ziel Maßzahlen als Ergebnis zu erhalten.
 


\section{Definitionen \& Begriffe}
\subsection{Structure from Motion}\label{sec:sfm}
Unter Structure from Motion, kurz SfM, 
zitat opencv dokument zu SfM
\subsection{Projektionsmatrix}
\subsection{Camera-Instrincs}
\subsection{Fundamental Matrix}
\subsection{Essential Matrix}
\subsection{Camera-Extrincs}
\subsection{Epipolar Geometry}

\section{Wie funktioniert Photogrammetry?}
