% !TeX root = ../../main.tex

\chapter{Stand der Technik}

\section{Was ist Photogrammetrie?}
Mit Photogrammetrie bezeichnet man Techniken, mit denen man Informationen wie Position, Orientierung, Form und Größe eines Objekts aus Bildern rekonstruieren kann.
Dabei unterscheidet man ob es sich bei den Bildern um photochemische (konventionelle Photographie), photoelektrische (digitale Photographie) oder um Laserscanner-Aufnahmen, dabei handelt es sich um Bilder welche zusätzlich Entfernungsinformationen zu jedem Bildelement enthalten, handelt.
Je nach Art der Weiterverarbeitung sowie Aufnahme der Bilder wird der Begriff der Photogrammetrie weiter differenziert.
Bei Weiterverarbeitung von photochemischen Bildern mittels optisch-mechanischer Geräte spricht man von \textbf{analoger Photogrammetrie}.
Bei Weiterverarbeitung von photochemischen Bildern mittels Computer spricht man von \textbf{analytischer Photogrammetrie}.
Bei Weiterverarbeitung von photoelektrischen Bildern mittels Computer, ein voll digitaler Prozess, spricht man von \textbf{digitaler Photogrammetrie}.
Die Ergebnisse einer Auswertung mittels Photogrammetrie können sein: \cite{kraus_2004}
\begin{itemize}
\item \textbf{Maßzahlen}, Koordinaten einzelner Objektpunkte in einem dreidimensionalen Koordinatensystem
\item \textbf{Zeichnungen}, Karten und Pläne im Grundriss mit Höhenlinien
\item \textbf{geometrische Modelle}
\item \textbf{Bilder}, entzerrte Fotos, Luftbildkarten
\end{itemize}

In dieser Arbeit werden Techniken der digitalen Photogrammetrie verwendet mit dem Ziel Maßzahlen als Ergebnis zu erhalten.
 


\section{Definitionen \& Begriffe}
\subsection{Structure from Motion}\label{sec:sfm}
Unter Structure from Motion, kurz SfM, 
zitat opencv dokument zu SfM

\subsection{Projektionsmatrix}

\subsection{Fundamental Matrix}
Die Fundamental Matrix ist eine algebraische Darstellung der Epipolargeometrie zwischen zwei Sichten.
Ziel dieser Geometrie ist die Suche von korrespondierenden Punkten in zwei Sichten.

Man nehme an, dass der Raumpunkt $X$, die Bildpunkte $x$ und $x'$, und die Kamerazentren $C$ und $C'$ in einer Ebene, genannt $\pi$, liegen (siehe Abb.~\ref{fig:theory-fundamental-matrix}).
Man erkennt, dass sich die Linien von den Kamerazentren durch $x$ bzw. $x'$ in $X$ schneiden und ebenfalls in der Ebene liegen.
Mit dieser Eigenschaft kann man die Suche von korrespondierenden Bildpunkten vereinfachen.
Nimmt man an, dass man nur den Bildpunkt $x$ kennt, dann kann man mit Hilfe der Ebene $\pi$ die Position von $x'$ einschränken.
Denn $x'$ muss sich auf der Schnittlinie der eben beschriebenen Ebene und der Bildebene der zweiten Sicht befinden. 
Diese Schnittlinie wird \emph{Epipolarlinie} genannt.
Die Punkte $e$ und $e'$ sind die Schnittpunkte des Vektors $\overline{CC'}$ mit den Bildebenen und werden Epipole genannt.
Die Ebene $\pi$ wird zusätzlich Epipolarebene genannt.
~\cite[Kapitel 9,1]{hartley_2000}

% S.240 / 241 multiple view geometry
Die Fundamental Matrix wird in~\cite[Kapitel 9.2]{hartley_2000} definiert als die $3\times 3$ Matrix, die für alle korrespondierenden Punkte $x \leftrightarrow x'$ 
\[ x'^TFx=0\] 
erfüllen.
Geometrisch betrachtet ist die F eine Abbildung der Punkte in einem Bild auf ihre Epipolarlinien im zweiten Bild.

\begin{figure}[h!]
    \centering
    \includegraphics{src/img/hartley_2000_fundamental_matrix.jpg}
    \caption{Veranschaulichung der Epipolargeometrie~\cite[Fig. 9.1]{hartley_2000}}
    \label{fig:theory-fundamental-matrix}
\end{figure}






\subsection{Camera-Instrincs}
\subsection{Fundamental Matrix}
\subsection{Essential Matrix}
\subsection{Camera-Extrincs}
\subsection{Epipolar Geometry}

\section{Wie funktioniert Photogrammetry?}
