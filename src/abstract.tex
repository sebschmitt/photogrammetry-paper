% !TeX root = ../main.tex

\addchap{Abstract}
\paragraph{Deutsch}\mbox{}\\
Diese Arbeit behandelt das Thema Structure from Motion, also das Rekonstruieren einer Struktur aus mehreren Bildern.
Im Rahmen dieser Arbeit wird ein Konzept einer Pipeline zum Rekonstruieren eines Objekt aus mehreren Bildern vorgestellt und implementiert.
Für die Implementierung wird C++, so wie die Bibliothek OpenCV verwendet.
Ziel des Projekts ist es, die Rekonstruktionspipeline so zu implementieren, dass keine Unkosten durch patentierte Algorithmen oder kostenpflichtiger Software und Tools entstehen.
Die Implementation wird beschrieben und theoretische Grundlagen werden im Voraus erläutert. 
Das Ergebnis des Projekts wird zum Schluss an Hand von vier Testfällen evaluiert.
Die Testfälle zeigen verscheiden Objekte mit unterschiedlichen Eigenschaften.
Die Ergebnisse der Testfälle werden zusammengefasst und es werden mögliche Fehlerursachen bei der Rekonstruktion näher erläutert.
Zum Schluss werden mögliche weitere Vorgehensweisen und Ansätze zur Behebung der Fehler beschrieben. 

\paragraph{English}\mbox{}\\
