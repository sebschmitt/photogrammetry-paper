% !TeX root = ../main.tex

\addchap{Abstract}
\paragraph{Deutsch}\mbox{}\\
Diese Arbeit behandelt das Thema Structure from Motion, also das Rekonstruieren einer Struktur aus mehreren Bildern.
Im Rahmen dieser Arbeit wird ein Konzept einer Pipeline zum Rekonstruieren eines Objekt aus mehreren Bildern vorgestellt und implementiert.
Für die Implementierung wird C++, so wie die Bibliothek OpenCV verwendet.
Ziel des Projekts ist es, die Rekonstruktionspipeline so zu implementieren, dass keine Unkosten durch patentierte Algorithmen oder kostenpflichtiger Software und Tools entstehen.
Die Implementation wird beschrieben und theoretische Grundlagen werden im Voraus erläutert. 
Das Ergebnis des Projekts wird zum Schluss an Hand von vier Testfällen evaluiert.
Die Testfälle zeigen verscheiden Objekte mit unterschiedlichen Eigenschaften.
Die Ergebnisse der Testfälle werden zusammengefasst und es werden mögliche Fehlerursachen bei der Rekonstruktion näher erläutert.
Zum Schluss werden mögliche weitere Vorgehensweisen und Ansätze zur Behebung der Fehler beschrieben. 

\paragraph{English}\mbox{}\\
This paper deals with the topic Structure from Motion, i.e. the reconstruction of a structure from several images.
In the context of this work, a concept for a pipeline for reconstruction of an object from several images is presented and implemented.
C++ is used for implementation along with the OpenCV library.
The aim of the project, is to implement the reconstruction pipeline in such a way that only free to use algorithms are used and there is no need to pay for algorithms, software or tools.
The implementation is described and theoretical basics are explained in advance.
The result of the project is finally evaluated using four test cases.
The test cases show different objects with different properties.
The results of the test cases are summarized and possible causes of errors in the reconstruction are explained in more detail.
At the end possible further procedures and approaches to correct the errors are described.