% !TeX root = ../../main.tex

\chapter{Testfälle}

In diesem Abschnitt wird die entwickelte Applikation in drei Testfällen getestet.
Die Testfälle sind dabei verschiedene Szenen mit mehreren Bildern von einem Objekt, das rekonstruiert werden soll.
Die Szenen und die Objekte sind dabei so gewählt, dass sie möglichst unterschiedlich sind in Bezug auf Größe und Beschaffenheit des Objekts.
In jedem Testfall wird die Szene kurz beschrieben und die Qualität der Feature Matches, sowie ihre rekonstruierten Punkte bewertet.
Bei der Bewertung wird zu einem auf die Beschaffenheit der Szene und des Objekts und zum anderem auf die Implementierung eingegangen.
In XXX werden die Ergebnisse der Testfälle zusammengefasst.

\section{Fall 1: Sessel}
Im ersten Testfall wird versucht einen Sessel zu rekonstruieren.
Dazu werden 10 Bilder verwendet, welche den Sessel aus verschiedenen Winkeln zeigt.
Die Bilder sind so sortiert, dass die Bewegung zwischen zwei Bildern minimal ist.
Die Anzahl der Matches, der rekonstruierten Punkte, sowie die Anzahl der Weltpunkte, die für die Skalierung genutzt wurden, so wie Skalierung selbst werden in \cref{tab:chair-results} gezeigt. % TODO: rewording
Das erste Bildpaar des Sessels wird in \cref{fig:chair-first-pair} gezeigt.
In \cref{fig:chair-first-pair-with-matches} wird das selbe Bildpaar mit allen Keypoints und ihren Matches gezeigt.
Matches sind dabei durch Linien gekennzeichnet, die die korrespondierenden Bildpunkte miteinander verbinden.
Hier fällt auf, dass sich die Matches überwiegend auf den Seiten des Sessels befinden.
Auch der Text auf der Decke und dem Kissen, so wie das karierte Muster des anderen Kissens werden als Keypoints erkannt und gematcht. % todo: rewording.
Des Weiteren sind viele Keypoints und Matches auf dem Teppich unten im Bild zu erkennen.
Im Gegensatz dazu gibt es sehr wenige Matches auf dem Boden, der Heizung, der Wand, sowie auf der Decke und dem Sitzkissen des Sessels.
Dies liegt daran, dass diese Texturen keine Kanten besitzen, die von den Feature Extractor als Feature erkannt werden können.
Dementsprechend fehlen diese Flächen bei den rekonstruierten Punkten, wie in \cref{fig:chair-model} zu sehen ist.  
Die generelle Form des Sessels und der Kissen ist in dem Modell gut zu erkennen.
Von der Decke ist nur ein streifen zu sehen, wo Matches an dem Text existieren.  
Das Sitzkissen ist überhaupt nicht zu sehen.
In \cref{fig:chair-model-2} ist das Modell von der anderen Seite und einem höheren Blickwinkel zu sehen.
Hier sind mehrere Punktsammlungen zusehen, die das gleiche darstellen, aber verschoben sind.
So etwa die Vorderseite und die Kissen des Sessels.
Dies lässt eine ungenaue Skalierung des Translationsvektors vermuten.

\begin{table}
    \begin{tabularx}{\textwidth}{c r r r r}
        \toprule
        Bildpaar &  Anzahl der Matches & Anzahl der Weltpunkte & Anzahl der Überlappende Weltpunkte & angewandte Skalierung \\ 
        \midrule
        1 & 1.570 & 1.569 & -  & - \\
        2 & 1.548 & 1.548 & 522 & 1,15455 \\
        3 & 1.405 & 1.404 & 494 & 1,09323 \\
        4 & 1.550 & 1.550 & 535 & 1,0002 \\
        5 & 1.426 & 1.413 & 658 & 0,796158 \\
        6 & 1.196 & 1.187 & 623 & 1,14341 \\
        7 & 923 & 918 & 372 & 0,523695 \\
        8 & 744 & 721 & 248 & 1,18605 \\
        9 & 1.352 & 1.351 & 241 & 0,851274 \\
        \midrule
        Summe & 11.714 & 11.661 & 3.693 & - \\
        \bottomrule
    \end{tabularx}
    \caption{XX}
    \label{tab:chair-results}
\end{table}

\begin{figure}
    \includegraphics[width=\textwidth]{src/img/chair_first_pair.jpg}
    \caption{Eins von zehn Bildpaaren zum Rekonstruieren des Sessels.}
    \label{fig:chair-first-pair}
\end{figure}

\begin{figure}
    \includegraphics[width=\textwidth]{src/img/chair_first_pair_with_matches.jpg}
    \caption{Bildpaar des Sessels mit Keypoints und Matches.}
    \label{fig:chair-first-pair-with-matches}
\end{figure}

\begin{figure}
    \includegraphics[width=\textwidth]{src/img/chair_model.jpg}
    \caption{Die rekonstruierten Bildpunkte des Sessels in Blender. Das Modell wurde um -119 Grad um die X gedreht, sowie um -4.8 Meter in Y und 2.7 Meter in Z Richtung verschoben.}
    \label{fig:chair-model}
\end{figure}

\begin{figure}
    \includegraphics[width=\textwidth]{src/img/chair_model_2.jpg}
    \caption{Das Sessel Modell in einem anderem Blickwinkel}
    \label{fig:chair-model-2}
\end{figure}


\section{Fall 2: Auto}
Im zweiten Testfall wird die Rekonstruktion eines Autos getestet. 
Insgesamt wurden 17 Bilder von einem schwarzen Skoda Fabia gemacht, die das Auto in verschiedenen Positionen zeigt.
Die Tabelle~\cref{tab:car-results} zeigt Anzahl der rekonstruierten Punkte pro Bildpaar.
Im Vergleich zum ersten Testfall gibt es hier deutlich weniger Matches, rekonstruierte Punkte und Punkte,die für die Skalierung verwendet worden sind.
\cref{fig:car-second-pair,fig:car-second-pair-with-matches} zeigen das zweite Bildpaar vom Auto.
Hier ist sehen, dass dass die Keypoints der meisten Matches nicht auf auf dem Auto liegen.
Statt dessen werden Matches eher an den Wänden, Pflanzen und auf dem Boden gefunden.
Bei dem Auto selbst werden die Matches hauptsächlich nur am Nummernschild und vereinzelt in den Reflexionen der Umgebung gefunden.
Wie durch die schlechten Matches zu erwarten ist, kann das Auto in den rekonstruierten Modell nicht erkannt werden, wie in \cref{fig:car-model} zu sehen ist.
In \cref{fig:car-model-2} ist das Modell aus einem anderem Winkel zu sehen.
Hier ist deutlich zu erkennen, dass die grünen Punkte der Pflanzen im Hintergrund in der gleichen Ebene wie viele Andere Punkte liegen. 
Dies lässt auf einen Fehler in der Triangulation oder Skalierung der Punkte vermuten.

\begin{table}
    \begin{tabularx}{\textwidth}{c r r r r}
        \toprule
        Bildpaar &  Anzahl der Matches & Anzahl der Weltpunkte & Anzahl der Überlappende Weltpunkte & angewandte Skalierung \\ 
        \midrule
        1  & 575 & 480 & -  & - \\
        2  & 910 & 860 & 165 & 0,468641 \\
        3  & 685 & 653 & 227 & 0,768563 \\
        4  & 583 & 570 & 51  & 0,788242 \\
        5  & 264 & 247 & 25  & 0,60277 \\
        6  & 427 & 409 & 58  & 0,848896 \\
        7  & 393 & 346 & 95  & 2,80038 \\
        8  & 799 & 527 & 99  & 0,923119 \\
        9  & 827 & 518 & 113 & 0,577807 \\
        10 & 618 & 597 & 125 & 1,11263 \\
        11 & 251 & 187 & 56  & 0,472208 \\
        12 & 68  & 36  & 10  & 1,49154 \\
        13 & 341 & 325 & 2   & 1,33112 \\
        14 & 334 & 263 & 18  & 2,02715 \\
        15 & 175 & 166 & 12  & 0,475839 \\
        16 & 167 & 146 & 21  & 0,651284 \\
        \midrule
        Summe & 7.417 & 6.330 & 1.077 & - \\
        \bottomrule
    \end{tabularx}
    \caption{XX}
    \label{tab:car-results}
\end{table}

\begin{figure}
    \includegraphics[width=\textwidth]{src/img/car_second_pair.jpg}
    \caption{Das zweite Bildpaar zum Rekonstruieren des Autos.}
    \label{fig:car-second-pair}
\end{figure}

\begin{figure}
    \includegraphics[width=\textwidth]{src/img/car_second_pair_with_matches.jpg}
    \caption{Das zweite Bildpaar vom Auto mit Keypoints und Matches.}
    \label{fig:car-second-pair-with-matches}
\end{figure}

\begin{figure}
    \includegraphics[width=\textwidth]{src/img/car_model.jpg}
    \caption{Das zweite Bildpaar vom Auto mit Keypoints und Matches.}
    \label{fig:car-model}
\end{figure}

\begin{figure}
    \includegraphics[width=\textwidth]{src/img/car_model.jpg}
    \caption{Das zweite Bildpaar vom Auto mit Keypoints und Matches.}
    \label{fig:car-model-2}
\end{figure}

\section{Fall 3: Hund}
Im dritten Testfall wird versucht mit fünf Bildern einen Hund in einem Körbchen zu rekonstruieren, der in \cref{fig:dog-image} zu sehen ist.

Bevor auf das Ergebnis der Rekonstruktion eingegangen wird, wird ein Problem beim Aufnehmen der Bilder erläutert.



Die \cref{tab:dog-results} zeigt die Ergebnisse der Rekonstruktion.
In allen Bildpaaren ist der Matches höher als im ersten Testfall.
Betracht man jedoch in XXX wo die Matches liegen, dann fällt auf, dass keine auf dem Hund liegen.
Statt dessen befinden sich alle Matches auf dem Teppich.
Das Modell in Blender zeigt dem entsprechend nur eine Fläche, die aus beige Punkten besteht.
Das Problem in diesem Testfall ist wie beim Auto und dem Sitzkissen des Sessels, dass der Hund keine auffälligen Kanten besitzt, die als Feature erkannt werden.

\begin{table}
    \begin{tabularx}{\textwidth}{c r r r r}
        \toprule
        Bildpaar &  Anzahl der Matches & Anzahl der Weltpunkte & Anzahl der Überlappende Weltpunkte & angewandte Skalierung \\ 
        \midrule
        1 & 4.799 & 4.786 & -  & - \\
        2 & 6.450 & 6.421 & 2.127 & 0,835131 \\
        3 & 6.525 & 6.472 & 2.763 & 0,892978 \\
        4 & 7.520 & 7.489 & 2.867 & 0,975344 \\
        4 & 7.462 & 7.444 & 3.215 & 0,801014 \\
        4 & 3.024 & 2.984 & 1.651 & 0,93856  \\
        4 & 6.071 & 6.009 & 1.337 & 1,42251  \\
        4 & 7.893 & 7.874 & 3.091 & 0,707064 \\
        4 & 6.917 & 4.350 & 1.924 & 7,01307  \\
        \midrule
        Summe & 11.714 & 11.661 & 3.693 & - \\
        \bottomrule
    \end{tabularx}
    \caption{XX}
    \label{tab:dog-results}
\end{table}

\begin{figure}
    \includegraphics[width=\textwidth]{src/img/dog.jpg}
    \caption{}
    \label{fig:dog-image}
\end{figure}


\begin{figure}
    \includegraphics[width=\textwidth]{src/img/dog_first_pair_with_matches.jpg}
    \caption{}
    \label{fig:dog-first-pair-with-matches}
\end{figure}

\begin{figure}
    \includegraphics[width=\textwidth]{src/img/dog_model.jpg}
    \caption{}
    \label{fig:dog-model}
\end{figure}

\begin{figure}
    \includegraphics[width=\textwidth]{src/img/dog_model_2.jpg}
    \caption{}
    \label{fig:dog-model-2}
\end{figure}
