% !TeX root = ../main.tex

\chapter{Ziele und Anforderungen}
Ziel der Arbeit ist es mit einfachen Mitteln eine Anwendung zur 3D-Rekonstruktion von Bildern zu entwickeln und die Machbarkeit davon zu bewerten.
Die Anwendung soll dabei aus einer Menge versetzter Bilder eines Objekts 3D Punkte des Objekts rekonstruieren und exportieren.
Dieses Vorgehen wird als Structure from Motion\footnote{Siehe dazu \autoref{sec:sfm} \nameref{sec:sfm}} bezeichnet.

Im folgenden Sind einzelne Ziele gesondert aufgeführt und erläutert.

\section{Ziele}
\begin{enumerate}
\item \label{goal:nocost} Für die Entwicklung und Nutzung der Software sollen keine Kosten entstehen
\item Bei der Entwicklung wird nach Möglichkeit auf offene Bibliotheken zurückgegriffen und möglichst wenig selbst implementiert
\end{enumerate}


\section{Anforderungen}
\begin{enumerate}
\item Es werden nur frei verfüg- und verwendbare Algorithmen verwendet um Ziel \autoref{goal:nocost} gerecht zu werden
\item  Die exportierten 3D Punkte lassen sich mit Blender\footnote{https://www.blender.org/} visualisieren
\item Die entwickelte Software soll unter aktuellen Versionen des Windows Betriebssystems lauffähig sein
\end{enumerate}
